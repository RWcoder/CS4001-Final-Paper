\documentclass[12pt, titlepage]{article}

\usepackage[margin=1in]{geometry}
\usepackage{microtype}
\usepackage{setspace}
\usepackage[backend=biber]{biblatex}
\addbibresource{sources.bib}

\doublespacing

\begin{document}

\title{Title goes here}
\author{David Thomson}
\maketitle

Arguably the most important change in society in the past few decades has been the automation of important tasks with computers, and the formation of a worldwide network of these computers that allows their storage and computing resources to be combined. It seems that almost every aspect of the modern world relies on technology in some way. This raises a natural question: if we can pay for our groceries with a self-checkout machine, cause a package to show up at our door by pressing the right combination of buttons on a computer, buy tickets at an electronic kiosk, and even deposit checks by taking a picture of them with a mobile phone, why is it that the main method of voting in important elections still involves going to a local building and filling out a piece of paper? Although it would be incredibly convenient to elect the president by sending a vote by text message or logging into a website at home, the unfortunate reality is that not all problems are easily solved with transistors. Given the current state of technology, electronic voting, especially online voting, is not secure enough to be considered a safe alternative to the traditional method of paper ballots.

The most common form of electronic voting today is the Direct Recording Electronic voting machine, or DRE. These are specialized computers with either a touchscreen or button interface that run a program that displays a ballot. The votes are counted by software on the machine and the results are stored on a memory card and printed on paper. In the 2012 U.S. presidential election, 39 percent of votes were cast with a DRE machine. \cite{kalb2015guide} It is imperative that a machine that has become such an integral part of our democracy be as secure as possible. Unfortunately, numerous studies have shown that DREs are frighteningly vulnerable to hacking and tampering.

A fundamental flaw with DREs is their verifiability. It is quite easy for any voter to understand how paper ballots work. This simplicity and transparency is incredibly important for voters to trust in the election. A DRE voting machine introduces a sea of black boxes that the average voter may not be able to comprehend. The electorate cannot be expected to simply trust that the manufacturers of DREs have made secure devices. Of course the software could be made open source, but what percentage of people would be able to read and understand the source code, and would take the time to do so? Furthermore, even releasing the source code and using a checksum to verify that that code is running on the actual machine does not necessarily prevent tampering with the hardware. It may be possible that the touchscreen itself could send incorrect touch coordinates to the program.

There are numerous examples of DREs being manipulated in several ways. In 2006 an independent study at Princeton showed the relative ease with which a standard DRE machine from the Diebold company could be compromised. \cite{feldman2006security} The machine installs software from a memory card. The slot for the card is covered with a metal panel that is kept in place with a lock. The study revealed that the same key was used for many different machines, and that the keys were easily copied. In addition, the   lock itself was cheap and able to be picked in seconds. This allowed the research team to insert a memory card with malicious software. The machine now functioned normally, except that a certain amount of votes were stolen from one candidate and given to another. The virus could tell whether or not the machine was in test mode, and only manipulated votes during the actual election. The software was able to modify every single piece of data relating to the final vote count, including the results stored in memory and the printed results, so there would be no way to tell that fraud had taken place. The team found that as long as the attacker was able to get their code onto the machine, they could make it quite sophisticated and include many parameters about how many votes to steal, how to calculate this number based on date and time, etc. 


The wide variety of these attacks attests to the increased complexity of an electronic voting system over a paper ballot system. The software, screen, and, memory components, as well as all the supply chains for these intricate parts introduce an incredible number of possible failure points.

Another requirement of an election is auditability. An election must be auditable so the results can be verified through a recount, but how can a recount be done if the votes are cast by pressing a screen? One common solution goes back to the reliability of paper. A Voter Verified Paper Audit Trail, or VVPAT, consists
%VVPAT hard to actual do the recount
%VVPAT same problems with verifiability 
%VVPAT can't stop a denial of service attack

\newpage
\printbibliography

\end{document}
